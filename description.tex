\documentclass{utap}

\usepackage{wrapfig}
\usepackage{xepersian}

\graphicspath{{./img/}}

\title{تمرین شماره‌ی ۷ - فاز اول}
\author{
	\href{mailto:bardia.eghbali@gmail.com?subject=[AP\%20S98\%20A7]\%20}{بردیا اقبالی},
	\href{mailto:ahhabibvand@gmail.com?subject=[AP\%20S98\%20A7]\%20}{امیرحسین حبیب‌وند},
	\href{mailto:farzadhabibii98@gmail.com?subject=[AP\%20S98\%20A7]\%20}{فرزاد حبیبی},
	\href{mailto:zangenehsaeed412@gmail.com?subject=[AP\%20S98\%20A7]\%20}{سعید زنگنه},
	\href{mailto:naghavi.pooya@gmail.com?subject=[AP\%20S98\%20A7]\%20}{پویا نقوی}
}
\course{برنامه‌سازی پیشرفته}
\lecturer{رامتین خسروی}
\deadline{جمعه ۲۷ اردیبهشت ۱۳۹۸، ساعت ۲۳:۵۵}

\begin{document}
	\maketitle
\section{IMDB}

	\subsection{نکات پایانی}
	\begin{itemize}
		\item
		\item
		\item
		\item
	\end{itemize}

\section{نحوه‌ی تحویل}
    پرونده‌‌های مربوط به برنامه‌ی خود را در پوشه‌ای با نام \lr{A7-Phase1-SID.zip} در صفحه‌ی \lr{CECM} درس بارگذاری کنید که \lr{SID} شماره‌ی دانشجویی شماست؛ برای مثال اگر شماره‌ی دانشجویی شما ۸۱۰۱۹۷۹۹۹ باشد، نام پوشه‌ی شما باید \lr{A7-Phase1-810197999.zip} باشد.
    \begin{itemize}
        \item
					برنامه‌ی شما باید در سیستم‌عامل لینوکس و با مترجم \lr{g++} با استاندارد \lr{\texttt{c++11}} ترجمه و اجرا شود.
				\item
					برنامه شما باید حتما طراحی شی‌گرا داشته باشد. همچنین باید به صورت \lr{Multifile} باشد و استفاده از \lr{Makefile} در این تمرین اجباری است.
				\item
					هدف این تمرین یادگیری شماست. لطفاً تمرین را خودتان انجام دهید. در صورت کشف تقلب مطابق قوانین درس با آن برخورد خواهد شد.
    \end{itemize}
\end{document}
