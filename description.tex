\documentclass{utap}

\usepackage{wrapfig}
\usepackage{verbatim}
\usepackage{fancyvrb}
\usepackage{lscape}
\usepackage{rotating}
\usepackage{xepersian}

\graphicspath{{./img/}}

\title{تمرین شماره‌ی ۷\normalsize\qquad فاز اول}
\author{%
	\href{mailto:bardia.eghbali@gmail.com?subject=[AP\%20S98\%20A7]\%20}{بردیا اقبالی},
	\href{mailto:ahhabibvand@gmail.com?subject=[AP\%20S98\%20A7]\%20}{امیرحسین حبیب‌وند},
	\href{mailto:farzadhabibii98@gmail.com?subject=[AP\%20S98\%20A7]\%20}{فرزاد حبیبی},
	\href{mailto:zangenehsaeed412@gmail.com?subject=[AP\%20S98\%20A7]\%20}{سعید زنگنه},
	\href{mailto:naghavi.pooya@gmail.com?subject=[AP\%20S98\%20A7]\%20}{پویا نقوی}%
}
\course{برنامه‌سازی پیشرفته}
\lecturer{رامتین خسروی}
\deadline{جمعه ۲۷ اردیبهشت ۱۳۹۸، ساعت ۲۳:۵۵}

\newcommand{\commandFormat}[2]
{
	\linespread{1.6}
	\begin{latin}
		\centering
		\begin{minipage}[t]{1\textwidth}
			{\VerbatimInput[frame=lines,label={\rl{دستور ورودی}}]{#1}}
			{\VerbatimInput[frame=lines,label={\rl{خروجی}}]{#2}}
		\end{minipage}
	\end{latin}
}


\begin{document}
	\maketitle
	\section*{IMDB}
		در پیاده‌سازی این تمرین باید تمامی آموخته‌های خود در درس را به کار گیرید.
		در این تمرین شما باید شبکەای مجازی پیادەسازی کنید که در آن افراد می‌توانند با دو نقش مختلف در آن ثبت‌نام کنند و از
		امکاناتی که در اختیارشان قرار داده می‌شود استفاده کنند. نقش اول ناشر است که می‌تواند در شبکەی مجازی فیلم‌های خود را قرار
		دهد تا بقیەی افراد بتوانند آنها را خریداری و مشاهده کنند. نقش دوم نیز مشتری‌ها هستند که می‌توانند به صورت جداگانه در
		شبکه ثبت‌نام کنند و به مشاهدە و امتیازدهی فیلم‌ها بپردازند.

		در فاز اول شما برنامەی این شبکەی مجازی را پیادەسازی می‌کنید که با استفاده از رابط کاربری خط فرمان می‌توان از آن استفاده
		کرد. در فازهای بعدی این برنامه گسترش می‌یابد


	\section{شرح تمرین}
	تعدادی دستور در ادامه تعریف می‌شود که شما ملزم به پیاده‌سازی آن‌ها هستید.
	تمامی دستورها پس از اجرا شدن دارای مشخصی دارند که برای هر دستور ذکر می‌شود.
	دقت کنید که خروجی‌های شما به صورت خودکار تست می‌شوند،
	لذا خروجی شما باید دقیقاً همانند خروجی خواسته شده باشد؛ در غیر این صورت نمره‌ی بخش آزمون
	\LTRfootnote{testcase}
	را از دست خواهید داد.


	\subsection{ثبت‌نام}
		با این دستور افراد می‌توانند در شبکه‌ی مجازی ثبت‌نام کنند. نام کاربری هرکاربر یک رشتەی یکتا در سیستم است
		\commandFormat{commands/signup.in.txt}{commands/signup.out.txt}

	\subsection{ورود}
		اگر فردی قبلاً در شبکه ثبت‌نام کرده باشد، پیش از استفاده از امکانات سامانه باید در آن وارد شود.
		توجه کنید که اگر کاربری همان لحظه در سامانه ثبت‌نام کند، نیازی به دستور ورود ندارد و پس از ثبت‌نام به صورت ضمنی با همان دستور وارد نیز شده است.
		پس از آن‌ که هر فرد به شبکه وارد شد، در همان ابتدا باید تمامی پیغام‌هایی که برای او ارسال شده و او هنوز آن‌ها را مشاهده نکرده است به نمایش در آید.
		پس از آن‌ تمامی این پیغام‌ها به لیست پیغام‌هایی که کاربر آن‌ها را خوانده اضافه خواهد شد.
		\commandFormat{commands/login.in.txt}{commands/login.out.txt}

	\subsection{ناشران}
		دو دستور قبل برای آن بود که کاربر‌ها بتوانند به منظور استفاده از امکانات به سامانه‌ وارد شوند.
		هر نوع از کاربران قابلیت‌های خاص خود دارند که از این پس قابلیت‌های آن‌ها را در قالب دستورهایی بیان می‌کنیم.

		\subsubsection{ثبت فیلم}
			هر ناشر می‌تواند با دستور زیر اطلاعات فیلم خود را وارد کرده و آن را به شبکه اضافه کند.
			قیمت فیلم‌ها به این دلیل است که هر کاربری که بخواهد این فیلم‌ را مشاهده کند باید معادل قیمت آن را پرداخت کند.
			\commandFormat{commands/addFilm.in.txt}{commands/addFilm.out.txt}

		\subsubsection{ویرایش اطلاعات فیلم}
			هر ناشر می‌تواند اطلاعات فیلم خود را ویرایش کند.
			\commandFormat{commands/editFilm.in.txt}{commands/editFilm.out.txt}

		\subsubsection{حذف فیلم}
			هر ناشر می‌تواند فیلمی را که خودش اضافه کرده است با دستور زیر از شبکه حذف کند.
			\commandFormat{commands/deleteFilm.in.txt}{commands/deleteFilm.out.txt}

		\subsubsection{مشاهده‌ی لیست فیلم‌ها}
			هر ناشر می‌تواند با دستور زیر فیلم‌هایی را که خودش در سایت قرار داده است مشاهده کند.
			\commandFormat{commands/getFilms.in.txt}{commands/getFilms.out.txt}

		\subsubsection{مشاهده جزییات فیلم}
			هر ناشر با دستور زیر می‌تواند جزییات فیلم خودش و نظراتی را که کاربران برای این فیلم ثبت کرده‌اند مشاهده کند.
			\commandFormat{commands/filmDetails.in.txt}{commands/filmDetails.out.txt}

		\subsubsection{پاسخ دادن به نظرات}
			هر ناشر می‌تواند به نظراتی که به فیلم‌هایش داده شده پاسخ دهد.
			\commandFormat{commands/commentReply.in.txt}{commands/commentReply.out.txt}

		\subsubsection{حذف کردن نظرات}
			هر ناشر می‌تواند نظراتی را که به فیلم‌هایش داده شده حذف کند.
			\commandFormat{commands/commentDelete.in.txt}{commands/commentDelete.out.txt}

		\subsubsection{مشاهده لیست دنبال‌کنندگان}
			هر ناشر می‌تواند لیست تمامی کسانی را که او را دنبال\LTRfootnote{follow} می‌کنند با دستور زیر مشاهده کند.
			\commandFormat{commands/publisherFollowers.in.txt}{commands/publisherFollowers.out.txt}

		\subsubsection{دریافت اعتبار}
			هر ناشر می‌تواند با دستور زیر اعتباری را که از فروش فیلم‌هایش به~دست~آورده است را از شبکه دریافت کند.
				هر فیلم در شبکه ممکن است یکی از سه حالت زیر را داشته باشه:
				\begin{enumerate}
					\item ضعیف: اگر امتیاز یک فیلم از ۵ کمتر باشد، این فیلم در حالت ضعیف قرار دارد.
					\item متوسط: اگر امتیاز یک فیلم بیشتر مساوی ۵ باشد و کمتر از ۸ باشد، این فیلم در حالت متوسط قرار دارد.
					\item قوی: اگر امتیاز یک فیلم بیشتر مساوی ۸ باشد، این فیلم در حالت قوی قرار دارد.
				\end{enumerate}
				وقتی کاربر پول پرداخت می‌کند، بر اساس اینکه فیلم در کدام یک از حالت‌های بالا باشد، مقداری از آن به ناشر تعلق می‌گیرد و باقی آن در حساب شبکه باقی می‌ماند.
				هر ناشر برای فیلم خود در حالت ضعیف باید۸۰ درصد، در حالت متوسط ۹۰ درصد و در حالت قوی ۹۵ درصد از مبلغ را از شبکه دریافت کند.
			\commandFormat{commands/getMoney.in.txt}{commands/getMoney.out.txt}

	\subsection{مشتریان}
		تا این جا با قابلیت‌هایی که تنها ناشر می‌تواند انجام دهد، آشنا شدیم.
		از این پس با قابلیت‌های مشتری‌ها آشنا می‌شویم.
		توجه کنید که هر ناشر برای فیلم‌های دیگر می‌تواند در نقش یک مشتری باشد.
		یعنی تمام قابلیت‌های مشتری را نیز علاوه بر قابلیت‌های ناشر داراست.

		\subsubsection{دنبال کردن}
			هر مشتری می‌تواند با دستور زیر ناشر مورد علاقه‌ی خود را دنبال کند.
			دنبال کردن ناشر این ویژگی را دارد که وقتی این ناشر فیلم جدیدی قرار می‌دهد به تمامی مشتری‌هایی که آن‌ را دنبال کرده‌اند، اطلاع‌رسانی می‌شود.
			همچنین هنگامی که کاربری یک ناشر را دنبال می‌کند باید برای آن ناشر پیغامی  ارسال شود.
			\commandFormat{commands/customerFollow.in.txt}{commands/customerFollow.out.txt}

		\subsubsection{افزایش اعتبار}
			هر مشتری می‌تواند حساب خود را شارژ کند و اعتبار خود را افزایش دهد.
			\commandFormat{commands/customerMoney.in.txt}{commands/customerMoney.out.txt}

		\subsubsection{جستجو}
			هر مشتری می‌تواند توسط دستور زیر بین تمامی فیلم‌ها جستجو انجام دهد.
			در صورتی که یک دشته برای جستجو دریافت ‌شود، باید در میان نام فیلم‌ها و نام کارگردان‌ها این رشته  را جستجو کند و نتیجه را نمایش دهد.
			در غیر این صورت باید ۱۰ فیلم جدیدی که به شبکه اضافه شده را نمایش دهد.
			\commandFormat{commands/search.in.txt}{commands/search.out.txt}

		\subsubsection{مشاهده جزییات فیلم}
			هر مشتری می‌تواند توسط دستور زیر یک فیلم را انتخاب کند و جزییات آن‌ را ببیند.
			همچنین در این قسمت پس از جزییات باید نظرهایی که برای آن فیلم گذاشته شده و پاسخ‌هایی که آن نظرات دریافت کرده‌اند را نمایش داده شود.
			در  انتهای این قسمت ۴ فیلم که بالاترین امتیاز را در میان فیلم‌های شبکه دارند پس از نمایش جزییات فیلم مذکور به نمایش در می‌آید.
			اگر امتیاز آن‌ها برابر بود بر اساس ترتیب اضافه شدن باید در خروجی چاپ شوند.
			توجه کنید که این فیلم‌ها نباید جزو فیلم‌هایی باشد که این کاربر قبلا خریداری کرده است.
			\commandFormat{commands/customerFilmDetails.in.txt}{commands/customerFilmDetails.out.txt}

		\subsubsection{خرید فیلم}
			هر مشتری می‌تواند هزینه‌ یک فیلم را پرداخت کند و آن را بخرد.
			این هزینه مستقیما به دارایی شبکه‌ی مجازی اضافه خواهد شد.
			\commandFormat{commands/buyFilm.in.txt}{commands/buyFilm.out.txt}

		\subsubsection{امتیازدهی به فیلم‌ها}
			هر مشتری می‌تواند به فیلم‌هایی که خریداری کرده است، توسط دستور زیر از بین ۱ تا ۱۰ به آن امتیاز دهد.
			امتیاز هر فیلم از میانگین امتیازاتی که مشتری‌ها به آن‌ها می‌دهند، محاسبه می‌شود.
			پس از هر امتیاز دادن پیغامی حاوی این امتیاز به ناشر آن فیلم ارسال می‌شود.
			\commandFormat{commands/rateFilm.in.txt}{commands/rateFilm.out.txt}

		\subsubsection{نظر دادن به فیلم‌ها}
			هر مشتری می‌تواند به فیلم‌هایی که خریداری کرده است، توسط دستور زیر نظر\LTRfootnote{comment} خود را برای فیلم ثبت کند..
			پس از هر نظر دادن پیغامی حاوی این نظر به ناشر ارسال می‌شود.
			\commandFormat{commands/customerComment.in.txt}{commands/customerComment.out.txt}

		\subsubsection{مشاهده لیست فیلم‌های خریداری شده}
			هر مشتری می‌تواند لیست فیلم‌هایی که خریداری کرده است را مشاهده کند.
			\commandFormat{commands/customerFilmList.in.txt}{commands/customerFilmList.out.txt}

		\subsubsection{مشاهده لیست پیغام‌ها}
			هر مشتری می‌تواند لیست پیغام‌هایی که برای او ارسال شده است را توسط این دستور مشاهده کند.
			توجه کنید که این پیغام‌ها باید به ترتیب زمان ارسال باشند و جدید‌ترین پیغام بالاتر باشد.
			\commandFormat{commands/customerNotifs.in.txt}{commands/customerNotifs.out.txt}

	\section{نکات پایانی}
		\begin{itemize}
			\item
			\item
			\item
			\item
		\end{itemize}

	\section{نحوه‌ی تحویل}
		پرونده‌‌های مربوط به برنامه‌ی خود را در پوشه‌ای با نام \lr{\path{A7-1-SID.zip}} در صفحه‌ی \lr{CECM} درس بارگذاری کنید که \lr{SID} شماره‌ی دانشجویی شماست؛ برای مثال اگر شماره‌ی دانشجویی شما ۸۱۰۱۹۷۹۹۹ باشد، نام پوشه‌ی شما باید \lr{\path{A7-Phase1-810197999.zip}} باشد.
		\begin{itemize}
			\item
						برنامه‌ی شما باید در سیستم‌عامل لینوکس و با مترجم \lr{g++} با استاندارد \lr{\texttt{c++11}} ترجمه و اجرا شود.
					\item
						برنامه‌ی شما باید حتما طراحی شیءگرا داشته باشد. همچنین باید به صورت \lr{Multifile} باشد و استفاده از \lr{Makefile} در این تمرین اجباری است.
					\item
						هدف این تمرین یادگیری شماست. لطفاً تمرین را خودتان انجام دهید. در صورت کشف تقلب مطابق قوانین درس با آن برخورد خواهد شد.
		\end{itemize}
\end{document}
